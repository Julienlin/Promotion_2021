% -*- ispell-local-dictionary: "francais"; ispell-check-comments: nil; -*-

\documentclass[a4]{article}
\usepackage{listings}

\usepackage{tikz} % Required for drawing custom shapes
\usetikzlibrary{shadows, arrows, decorations.pathmorphing, fadings, shapes.arrows, positioning, calc, shapes, fit, matrix}

\usepackage{polyglossia}

\definecolor{lightblue}{RGB}{0,200,255} 
\definecolor{paper}{RGB}{245,237,187}
\definecolor{ocre}{RGB}{243,102,25} % Define the orange color used for highlighting throughout the book
\definecolor{BurntOrange}{RGB}{238,154,0}
\definecolor{OliveGreen}{RGB}{188,238,104}
\definecolor{DarkGreen}{RGB}{0,128,0}
\definecolor{BrickRed}{RGB}{238,44,44}
\definecolor{Tan}{RGB}{210,180,140}
\definecolor{Aquamarine}{RGB}{127,255,212}
\definecolor{NavyBlue}{RGB}{0,64,128}

\lstdefinestyle{customcpp}{
    breaklines=true,
    frame=shadow,
    xleftmargin=\parindent,
    language=C++,
    showstringspaces=false,
    basicstyle=\tiny\ttfamily,
    keywordstyle=\bfseries\color{green!40!black},
    commentstyle=\itshape\color{purple!40!black},
    identifierstyle=\color{blue},
    stringstyle=\color{orange!60!black},
}

\newcommand*\lstinputpath[1]{\lstset{inputpath=#1}}
\lstinputpath{./exemples/}
\newcommand{\includepartcode}[4][cpp]{
\lstinputlisting[escapechar=, firstline=#3, lastline=#4, style=custom#1]{#2}
}

\title{Optimisation par colonies de fourmis}
\lstset{language=C++,
  frame=single,
  backgroundcolor=\color{paper},
  basicstyle=\tiny\ttfamily,
  keywordstyle=\color{blue}\tiny\ttfamily,
  commentstyle=\color{red}\tiny\ttfamily,
  stringstyle=\color{brown}\tiny\ttfamily,
  keepspaces=true,
  showspaces=false,
  showstringspaces=false,
  tabsize=4
}


\newcommand{\avg}{\textrm{avg}}


\author{X.Juvigny}
\begin{document}
\maketitle

\section{Les algorithmes coopératifs}

Le terme ``\textsl{Intelligence en essaim}'' a été introduit en 1989 par Beni et al. Cette classe d'algorithmes, souvent inspirés par le comportement des insectes sociaux, met en place une population d'agents simples interagissant et communicant indirectement avec leur
environnement. Ces algorithmes constituent une classe d'algorithmes massivement parallèles pour résoudre une tâche donnée.

L'algorithme en essaim le plus connu est l'optimisation par colonies de fourmis (ACO) pour les problèmes combinatoires. Dans ce projet,
on revient à l'inspiration originale des algorithmes ACO où une population d'agents simples (qui peuvent être vus comme imitant le
comportement de fourmis réelles) résout efficacement le problème de fourragement (chercher le chemin le plus court de la fourmilière
à une source de nourriture).    % fourrageage ou approvisionnement  selon https://fr.wikipedia.org/wiki/Fourrageage

\section{Modèle simple d'ACO}
Dans cette partie, nous allons décrire un modèle simple de colonies de fourmis qui permet de résoudre le problème de fourragement.
Il a été montré pour cet algorithme qu'il converge de façon sur-linéaire par rapport au nombre de fourmis.

\subsection{Description du modèle}

On considère un terrain représenté par une grille cartésienne 2D, dont chaque cellule comporte une valeur donnant l'unité de temps pour
la traverser (valeur comprise entre zéro et un). 
On considère un ensemble de $m$ fourmis artificielles qui évoluent sur ces cellules et sur lesquelles elles mettent à jour des ``taux de
phéromones''. Chaque cellule $s$ de la grille stocke en plus deux valeurs réelles, correspondant à deux types de phéromones : un phéromone
$V_{1}(s)$ permettant d'indiquer aux autres fourmis le chemin d'exploration effectué, et un phéromone $V_{2}(s)$ permettant à la fourmi
de pouvoir retourner au nid lorsqu'elle a trouvé de la nourriture. 

Sur la grille, on trouve quatre types de cellules : une cellule correspondant à la fourmilière, une cellule correspondant à la source de
nourriture (qui pourrait ne pas être unique), des cellules indésirables que le fourmis ne peuvent pas traverser, et le reste des
cellules sont \textsl{libres}, c'est à dire explorables par les fourmis.

Une fourmi peut être dans deux états possibles : elle peut porter de la nourriture (état ``chargée'') ou elle peut ne rien porter (état ``non chargée''). En évoluant sur la grille, l'état des fourmis peut changer selon les règles naturelles suivantes :
\begin{itemize}
\item si une fourmi arrive sur une cellule contenant de la nourriture, son état devient ``chargée'',
\item et si elle arrive à la fourmilière son état change à ``non chargée'' ;
\item quand une fourmi arrive à la fourmilière avec l'état chargée, un compteur ``d'unités de nourriture'' est incrémenté.
Ce compteur servira d'indice de performance globale pour la population de fourmi choisie.
\end{itemize}

Décrivons la dynamique du modèle. Au départ :
\begin{itemize}
\item Le compteur d'unités de nourriture est mis à zéro ;
\item les $m$ fourmis sont initialisées à des positions arbitraires
  (soit toutes dans la fourmilière, soit initialisées uniformément sur la grille) ;
\item toutes les fourmis sont non chargées ;
\item les phéromones sont toutes mises à zéro sur la grille.
\end{itemize}

À chaque pas de temps, une fourmi fait deux choses :
\begin{enumerate}
\item Elle met à jour les taux de phéromones $V_{1}(s)$ et $V_{2}(s)$ de la cellule sur laquelle elle se trouve en utilisant celles des quatre cellules voisines (on note $N(s)$ les voisins de $s$).
  La mise à jour des taux de phéromones requiert uniquement la connaissance du maximum et de la moyenne des cellules voisines :
  $\max_{i}\left(N(s)\right) \equiv \max_{s'\in N(s)}V_{i}(s')$
  et $\avg_{i}\left(N(s)\right)\equiv\frac{1}{4} \sum_{s'\in N(s)} V_{i}(s')$
  où l'indice $i\in\left\{1,2\right\}$ indique le type de phéromone considéré.
  Précisément, la mise à jour des phéromones se fait selon les calculs suivants :
\[
V_{1}(s) \rightarrow \left\{\begin{array}{ll} 
 1 & \mbox{si la source de nourriture est en }s \\
 \alpha \max_{1}(N(s)) + (1-\alpha)\avg_{1}(s) & sinon
 \end{array}\right.
 \]
 et
\[
V_{2}(s) \rightarrow \left\{\begin{array}{ll} 
 1 & \mbox{si la fourmilière est en }s \\
 \alpha \max_{2}(N(s)) + (1-\alpha)\avg_{2}(s) & sinon
 \end{array}\right.
 \]
où $0\leq \alpha \leq 1$ 
\item elle avance sur une de ses cellules voisines si elle n'a pas la valeur -1 :
\begin{itemize}
\item avec une probabilité $\varepsilon$ ($0 \leq \varepsilon \leq 1$) qu'on appellera \textsl{taux d'exploration}, elle avancera sur une cellule voisine choisie aléatoirement parmi ses quatre voisines;
\item avec une probabilité $1-\varepsilon$, elle avancera sur la cellule ayant le taux de $V_{1}(s)$ le plus grand si elle n'est pas
chargée, et sur la cellule ayant le taux de $V_{2}(s)$ le plus grand si elle est chargée.
\end{itemize}
Le paramètre $\alpha$ est appelé paramètre de bruit.
\item la fourmi avancera dans le même pas de temps d'autant qu'elle n'a pas épuisé ses points de mouvement : à chaque pas de temps, la
fourmi possède un taux de mouvement de 1 qu'elle doit dépenser en fonction de la valeur unité de temps du terrain qu'elle traverse (ce
qui traduit la difficulté plus ou moins grande de traverser une zone). 
\end{enumerate}

À chaque pas de temps, les phéromones posées par les fourmis s'évaporent suivant un cœfficient d'évaporation $\beta$ :
$V_{i}(s) \rightarrow \beta V_{i}(s)$. La valeur $\beta$ prend typiquement une valeur proche de un. 

Ainsi, pour l'instanciation du modèle, les paramètres suivants doivent être précisés :
\begin{itemize}
\item \textbf{l'environnement} : c'est l'ensemble des cellules avec leur unité de temps, leur type (libre, indésirable, fourmilière ou nourriture)
\item \textbf{le nombre de fourmis} $m$
\item l'initialisation de la position des fourmis : soit toutes au départ dans la fourmilière soit réparties sur la grille uniformément ;
\item les paramètres de bruit $\alpha$, d'évaporation $\beta$ et d'exploration $\varepsilon$.
\end{itemize}

Le modèle décrit ci-dessus est constitué d'agents réactifs simples qui communiquent indirectement au travers de l'environnement. Contrairement aux modèles classiques, il n'est pas nécessaire ici de conserver le trajet d'une fourmi par rapport à sa fourmilière grâce à l'utilisation d'une deuxième phéromone, ce qui rend nos agents complètement réactifs.

\subsection{Création de l'environnement}

Notre simulation nécessite l'utilisation d'une carte d'unités de temps traduisant la nature plus ou moins accidentée du terrain traversé
par les fourmis (se traduisant par une valeur de temps que doit dépenser une fourmi pour rentrer dans une cellule). La génération d'une telle
carte peut se faire soit de façon analytique (à l'aide d'une fonction) soit de façon stochastique en générant un plasma.

C'est la deuxième solution qui a été retenue pour ce projet. Un plasma est une carte générée de façon aléatoire où on s'assure que
le gradient de valeurs entre deux points de la carte ne dépasse pas une certaine valeur $d$ nommée déviation.

On considère une grille composée de $n\times n$ sous--grilles. Ces sous--grilles ont un nombre de cellules par direction de
$ns=2^{k}+1$ cellules et se recouvrent avec les sous--grilles voisines à l'aide d'une rangée de cellule.
L'algorithme est alors le suivant :
\begin{itemize}
\item On génère les coins de chaque sous--grille en générant aléatoirement le premier coin de la première sous--grille puis en générant les autres coins séquentiellement à partir de ce premier coin en calculant une déviation inférieure à $d\times ns$.
\item Puis on divise récursivement chaque sous--grille en quatre sous--grilles égales en calculant la valeur des cellules se trouvant au milieu
de chaque bord et la valeur de la cellule se trouvant au centre de la sous-grille, en calculant des déviations inférieures à
$d\times \frac{ns}{2}$. Il faudra tout de même veiller à chaque instant aux bords, communs à chaque sous--grille
\end{itemize}

Une fois toutes les valeurs générées, on les normalise pour obtenir une carte de valeurs comprises entre zéro et un.

\section{Parallélisation du code}

Il existe deux façons de paralléliser le code (en fait trois, mais la troisième est beaucoup plus complexe à réaliser) :

\subsection{Première façon}
L'idée ici est que chaque processus contienne l'environnement en entier et ne contrôle qu'une partie des fourmis tandis que lors de la mise
à jour des phéromones lors de la phase d'évaporation, chaque processus s'occupe d'une partie de la carte pour laquelle il calculera l'évaporation, à l'aide d'une parallélisation OpenMP si besoin.

Lorsque deux fourmis appartenant à deux processus différents se trouvent sur une même cellule, la valeur de la case va dépendre des valeurs des phéromones des cellules voisines, qui peuvent être différentes selon si d'autres fourmis locales sont passées avant ou non ! Cependant, l'algorithme peut normalement prendre les fourmis dans un ordre arbitraire, et donc dans ce cas, on choisira de prendre la valeur la plus grande d'entre tous les processus comme valeur de phéromone pour une cellule donnée !

Le problème de cette approche est qu'elle n'est efficace que si le nombre de fourmi est grand et la carte assez petite pour tenir en mémoire sur chaque processus. De plus, l'évaporation des phéromones se fait en parallèle, assurant ainsi un bon degré de parallélisme. De plus, cette méthode assure un bon équilibrage des tâches.

Par contre, une grande quantité de données est échangée entre les processus.

\subsection{Seconde façon}

Cette fois-ci, chaque processus ne prend en compte qu'une partie de la carte et ne gère que les fourmis qui sont sur sa sous-carte. La difficulté ici est de gérer les bords de chaque sous-carte. De plus, il est possible qu'un processus n'ait pas de fourmis à gérer tandis qu'un autre en possède beaucoup (en particulier celui possédant la fourmilière). L'équilibre des tâches ne sera donc pas optimal, mais si le nombre de fourmis est assez grand ainsi que le coefficient d'exploration, il est très probable que les fourmis soient relativement bien distribuées sur la carte.

L'avantage de cette méthode est que la mémoire occupée par l'application est bien plus petite qu'à l'aide de la première solution. De plus,
il n'y a qu'un échange au bord des sous-cartes et donc peu de données échangées en définitive.

\end{document}
