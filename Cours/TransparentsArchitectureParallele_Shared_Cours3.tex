\documentclass[handout,francais]{beamer}
\usetheme{CambridgeUS}
%\input{header}
%\usepackage[utf8x]{inputenc}
\usepackage{amsmath,amsfonts,graphicx}
%\usepackage{beamerthemeONERA}
\usepackage{listings}
\usepackage{dsfont}
\usepackage{enumitem}
%\setlist[description]{style = multiline, labelwidth = 60pt}

\usepackage{tikz} % Required for drawing custom shapes
\usetikzlibrary{shadows, arrows, decorations.pathmorphing, fadings, shapes.arrows, positioning, calc, shapes, fit, matrix}

\usepackage{polyglossia}
\usepackage{multirow}
\usepackage{colortbl}

\institute
[ONERA, DTIM/CHP]
{Office National d'Etudes et de Recherches Aérospatiales,\\
\inst{1}Département Aéroacoustique}


\definecolor{lightblue}{RGB}{0,200,255} 
\definecolor{paper}{RGB}{239,227,157}
\definecolor{ocre}{RGB}{243,102,25} % Define the orange color used for highlighting throughout the book
\definecolor{BurntOrange}{RGB}{238,154,0}
\definecolor{OliveGreen}{RGB}{188,238,104}
\definecolor{DarkGreen}{RGB}{0,128,0}
\definecolor{BrickRed}{RGB}{238,44,44}
\definecolor{Tan}{RGB}{210,180,140}
\definecolor{Aquamarine}{RGB}{127,255,212}
\definecolor{NavyBlue}{RGB}{0,64,128}

\title[Calculateurs parallèles à mémoire partagée\hspace{2em}]{Calculateurs à mémoire partagée}
\author[Xavier JUVIGNY]{Xavier JUVIGNY}
\date{\today}

\institute{ONERA}
\usepackage{enumitem}
\setitemize{label=\usebeamerfont*{itemize item}%
            \usebeamercolor[fg]{itemize item}
            \usebeamertemplate{itemize item}}

\begin{document}

\lstset{%
  basicstyle=\scriptsize,
  frame=single,
  keywordstyle=\scriptsize\color{blue},
  language=C++,
  xleftmargin=\parindent,
  commentstyle=\tiny\ttfamily\color{red},
  stringstyle=\scriptsize\color{brown},
  keepspaces=true,
  showspaces=false,
  tabsize=2
}

\begin{frame}
 \titlepage
\end{frame}

\begin{frame}
\frametitle{Plan du cours}
\tableofcontents
\end{frame}

\section{Programmation sur machine à mémoire partagée}

\subsection{Notion de thread}

\begin{frame}{Processus}

\begin{block}{Définition}
\begin{itemize}
\item Créée par l'OS ou un exécutable
\item Contient beaucoup d'informations propres :
\begin{itemize}
\item id processus, user id, group id
\item environnement, répertoire de travail
\item instruction du programme, registre, tas, pile, fichiers
\item bibliothèques partagées, signaux action, etc.
\end{itemize}
\item Prends beaucoup de ressources à sa création
\end{itemize}
\end{block}
\end{frame}

\begin{frame}{Thread}

{\scriptsize
\begin{block}{Définition}
\begin{itemize}
\item Exécutable léger créer par un processus;
\item Contient peu d'information propre :
\begin{itemize}
\item tas, registres, politique d'exécution
\item variables propres au code exécuté par le thread;
\end{itemize}
\item Possède son propre flot d'instructions;
\item Partage ressources processus père avec d'autres threads
\item Meurt avec son processus parent;
\end{itemize}
\end{block}

\begin{block}{Propriétés mémoire partagée}
\begin{itemize}
\item Changement ressource du père par thread vu par tous ( fermeture fichier,\ldots );
\item Deux threads ont deux pointeurs égaux $\Rightarrow$  la même donnée;
\item Conflits mémoires possible lors d'accès en écriture par deux threads sur une variable partagée;
\end{itemize}
\end{block}
}
\end{frame}

\begin{frame}[fragile]{Comparaison performance création processus vs threads}

{\scriptsize
\begin{table}[h]
\begin{center}
 \begin{tabular}{|c||c|c|c||c|c|c|}\hline
 \multirow{2}{*}
 {\bf Plateforme} & \multicolumn{3}{c||}{\lstinline@fork()@} & \multicolumn{3}{c|}{\lstinline@std::thread@} \\ \cline{2-7}
 & real & user & sys & real & user & sys \\ \hline \hline
 \rowcolor{lightgray!25}Intel 2.6Ghz Xeon E5-2670 ( 16 c{\oe}urs/n{\oe}uds ) & 8.1 & 0.1 & 2.9 & 0.9 & 0.2 & 0.3 \\ \hline
 Intel 2.8GHz Xeon 5660 ( 12 c{\oe}urs/n{\oe}uds ) & 4.4 & 0.4 & 4.3 & 0.7 & 0.2 & 0.5 \\ \hline
 \rowcolor{lightgray!25}AMD   2.4GHz Opteron ( 8 c{\oe}urs/n{\oe}uds ) & 17.6 & 2.2 & 15.7 & 1.4 & 0.3 & 1.3 \\ \hline
 IBM 4.0 GHz Power 6 ( 8 cpus/n{\oe}uds ) & 9.5 & 0.6 & 8.8 & 1.6 & 0.1 & 0.4 \\ \hline
 \end{tabular}
\end{center}
 \caption{Comparaison création de 50 000 processus ( fork() ) contre création de 50 000 threads ( std::thread )}
 \label{tab:pvst}
\end{table}
}

\end{frame}

\begin{frame}[fragile]{Création et terminaison de threads}

\begin{figure}[h]
\begin{center}
\begin{tikzpicture}
\node[rectangle, fill=yellow,draw] (P) {Processus};
\node[below = 6cm of P.south] (E){};
\draw[dashed,thick,red,-latex] (P) -- (E);

\node[below = 7mm of P.south] (N11) {};
\node[below = 1cm of P.south] (N1) {};
\node[right = 2cm of N1.south east, rectangle, fill=cyan, draw] (T1) {Thread};
\draw[dashed, blue, -latex] (N11.north) -|node[above]{\scriptsize Création} (T1.north);
\node[below = 55mm of P.south] (TE1){};
\draw[dashed, blue, -latex] (T1.south) |-node[below]{\scriptsize Synchronisation et destruction} (TE1.south);

\node[below = 1.7cm of P.south] (N21) {};
\node[below = 2cm of P.south] (N2) {};
\node[left = 2cm of N2.south west, rectangle, fill=green!40, draw] (T2) {Thread};
\draw[dashed, green!20!black, -latex] (N21.north) -|node[above]{\scriptsize Création} (T2.north);
\node[below = 45mm of P.south] (TE2){};
\draw[dashed, green!20!black, -latex] (T2.south) |-node[below]{\scriptsize Synchronisation et destruction} (TE2.south);

\end{tikzpicture}
\end{center}
\caption{Exemple de création, synchronisation et terminaison de thread}
\end{figure}

\end{frame}

\begin{frame}[fragile]{Ajustement du nombre de threads et Hyperthreading}

\begin{block}{Définition de l'hyperthreading}
\begin{itemize}
 \item Deux fois plus de décodeur d'instruction que d'unités de calcul;
 \item Deux threads sur une unité de calcul peuvent se partager les ressources
       de cette unité ( par exemple calcul sur les entiers et calcul sur les réels ).
\end{itemize}
\end{block}


\begin{block}{Performance de l'hyperthreading}
\begin{itemize}
 \item Au mieux, en prenant deux fois plus de threads que d'unité de calcul, une accélération de 30\%
 par rapport à un calcul fait avec un nombre de thread égal au nombre d'unité de calcul.
\end{itemize}

\end{block}

\alert{Pour connaître le nombre d'unité de calcul, attention aux informations données par vos OS. En général, le nombre de CPUs affichés
est en fait le nombre de décodeur d'instructions. Pour connaître le nombre d'unité de calcul, chercher le nombre de c{\oe}urs disponibles.}

\end{frame}

\begin{frame}[fragile]{Exemple de sortie de lscpu}

\texttt{
Architecture:          x86\_64\\
Mode(s) opératoire(s) des processeurs :32-bit, 64-bit\\
Byte Order:            Little Endian\\
\textcolor{red}{CPU(s):                4}\\
On-line CPU(s) list:   0-3\\
\textcolor{red}{Thread(s) par cœur : 2}\\
\textcolor{blue}{Cœur(s) par socket : 2}\\
\textcolor{blue}{Socket(s):             1}\\
Nœud(s) NUMA :       1\\
Identifiant constructeur :GenuineIntel\\
Famille de processeur :6\\
Modèle :             42\\
Model name:            Intel(R) Core(TM) i7-2620M CPU \@ 2.70GHz\\
Révision :           7\\
...
}
\end{frame}

\subsection{Conflits d'accès aux données}

\begin{frame}[fragile]{Concurrence d'accès aux données}
 \begin{block}{Définition}
  On dit que des threads sont en concurrence d'accès aux données
  (Datarace condition) quand :
  \begin{itemize}
   \item Un thread lit une variable partagée ``pendant'' qu'un deuxième thread la modifie;
   \item Les deux threads tentent de modifier ``simultanément'' la même variable.
  \end{itemize}
 \end{block}

 \begin{block}{Phénémologie}
  \begin{itemize}
   \item Valeurs ``aléatoires'', voire incohérentes, de certaines variables partagées;
   \item Plantage aléatoire du code sur des accès mémoires;
  \end{itemize}

 \end{block}

\end{frame}

\begin{frame}[fragile]{Section séquentielle}

\begin{block}{Cas de sections séquentielles}
 \begin{itemize}
  \item On garde une ``zone'' du programme afin que les threads ne puissent passer
  que un par un ou qu'un seul thread y passe;
  \item Cette zone peut être complexe, comme modifier les valeurs d'un tableau, modifier un
  dictionnaire, une liste, une pile, etc.
  \item Généralement, mécanisme relativement lourd et lent.
  \item Exemples : mutex en pthreads, critical en OpenMP, etc.
 \end{itemize}

\end{block}

\end{frame}

\begin{frame}[fragile]{Opération atomique}
 \begin{block}{Cas des opérations atomiques}
  \begin{itemize}
   \item La section séquentielle se réduit à une seule opération simple sur un type de base ( entier, réels ,etc. )
   \item Mécanisme léger et non pénalisant;
   \item Exemples : atomic sur pthread et OpenMP.
  \end{itemize}

 \end{block}

\end{frame}

\subsection{Optimisation d'applications parallèles en mémoire partagée}

\begin{frame}[fragile]{Affinité}
 
 \begin{block}{Définition}
  Le système d'exploitation attache un thread à une unité de calcul spécifique : on parle d'affinité
 \end{block}

 \begin{block}{Intérêt}
  \begin{itemize}
   \item Par défaut, un thread s'exécute sur n'importe quel unité de calcul, et peut migrer durant son exécution;
   \item Lorsqu'il migre, on pert les données en mémoire cache;
   \item Sur architecture NUMA, on peut même changer de socket !
   \item On peut contrôler l'affinité avec l'utilitaire \verb@taskset@ (Linux) sinon API différent selon bibliothèque et
   OS utilisé !
  \end{itemize}

 \end{block}
 
\end{frame}

\begin{frame}[fragile]{Application ``Memory bound''}
 
 \begin{block}{Définition}
  \begin{itemize}
   \item Quand la vitesse d'exécution de l'application limitée par la vitesse d'accès aux données
  \end{itemize}
 \end{block}

 \begin{block}{Explication}
  \begin{itemize}
   \item La mémoire vive plus lente que le traitement des données par un CPU;
   \item Mémoire cache et mémoire interlacée permet d'accélérer l'accès aux données;
   \item Algorithmes pas toujours possible d'optimiser en mémoire cache ou avec mémoire interlacée;
   \item Plus le nombre d'unités de calcul grand, plus difficile de ne pas être memory bound.
  \end{itemize}

 \end{block}

 
\end{frame}

\begin{frame}[fragile]{Application ``cpu bound''}
 
\begin{block}{Définition}
 \begin{itemize}
  \item Quand la vitesse d'exécution de l'application est limitée à la vitesse de traitement du CPU
 \end{itemize}
\end{block}

\begin{block}{Explication}
 \begin{itemize}
  \item Programme bien optimisé pour l'architecture mémoire du calculteur;
  \item Toujours chercher à être le plus ``cpu bound''.
  \item Pas toujours possible !
 \end{itemize}

\end{block}

\end{frame}

\begin{frame}[fragile]{Quand peut-on être cpu bound ?}
 
  \begin{block}{Avec la mémoire cache}
 \begin{itemize}
  \item Quand le nombre de données accédées $<$ nombre d'opérations effectuées;
  \item Pas de saut mémoire non réguliers pour la preamption de cache;
  \item Avoir un algorithme local en temps et espace : accèder dans le délai le plus bref aux mêmes données
  le plus possible.
 \end{itemize}
 \end{block}
 
 \begin{block}{Avec la mémoire interlacée}
  \begin{itemize}
   \item Quand on accède à des données successives sur des bancs mémoires différents;
   \item Ou accès à très peu de variables différentes pour stocker sur des registres.
  \end{itemize}

 \end{block}

\end{frame}

\begin{frame}[fragile]{Exemple suite itérative}
 \begin{lstlisting}
double u = 56547;
for ( unsigned int iter = 0; iter < 1023; ++iter ) {
    u = (u%2 == 0 ? u/2 : (3*u+1)/2);
}
 \end{lstlisting}

 \begin{columns}[t]
  \begin{column}{0.5\textwidth}
   \textcolor{green!25!black}{Mémoire cache} 
   \\ 
   \textcolor{blue}{cpu bound}
   \end{column}
   \begin{column}{0.5\textwidth}
   \textcolor{orange!25!black}{Mémoire interlacée}
   \\
   \textcolor{blue}{cpu bound}
  \end{column}

 \end{columns}
 
\end{frame}

\begin{frame}[fragile]{Exemple opération vectorielle}
 \begin{lstlisting}
  unsigned long N = 1000000UL;
  std::vector<double> u(N), v(N), w(N);
  for ( int i = 0; i < N; ++i ) {
      u[i] = std::max(v[i],w[i]);
  }
 \end{lstlisting}

 \begin{columns}[t]
  \begin{column}{0.5\textwidth}
   \textcolor{green!25!black}{Mémoire cache} 
   \\ 
   \textcolor{red}{memory bound}
   \end{column}
   \begin{column}{0.5\textwidth}
   \textcolor{orange!25!black}{Mémoire interlacée}
   \\
   \textcolor{blue}{cpu bound}
  \end{column}
 \end{columns} 
\end{frame}

\begin{frame}[fragile]{Opération matricielle}
 \begin{lstlisting}
std::vector<double> A(N*N);
std::vector<double> B(N*N);
std::vector<double> C(N*N);
...
for ( int i = 0; i < N; ++i )
  for ( int j = 0; j < N; ++j )
    for ( int k = 0; k < N; ++k )
      C[i+j*N] += A[i+k*N]*B[k+j*N];
 \end{lstlisting}

 \begin{columns}[t]
  \begin{column}{0.5\textwidth}
   \textcolor{green!25!black}{Mémoire cache} 
   \\ 
   \textcolor{red}{memory bound}
   \\
   Le code peut être optimiser !
   \end{column}
   \begin{column}{0.5\textwidth}
   \textcolor{orange!25!black}{Mémoire interlacée}
   \\
   \textcolor{blue}{memory bound}
   \\
   Le code peut être optimiser !
  \end{column}
 \end{columns} 

\end{frame}

\begin{frame}[fragile]{Opération matricielle ( suite )}
 \begin{lstlisting}
std::vector<double> A(N*N);
std::vector<double> B(N*N);
std::vector<double> C(N*N);
...
for ( int k = 0; k < N; ++k )
  for ( int j = 0; j < N; ++j )
    for ( int i = 0; i < N; ++i )
      C[i+j*N] += A[i+k*N]*B[k+j*N];
\end{lstlisting}

 \begin{columns}[t]
  \begin{column}{0.5\textwidth}
   \textcolor{green!25!black}{Mémoire cache} 
   \\ 
   \textcolor{red}{memory bound/cpu bound}
   \\
   Mieux mais peut être encore optimiser !
   \end{column}
   \begin{column}{0.5\textwidth}
   \textcolor{orange!25!black}{Mémoire interlacée}
   \\
   \textcolor{blue}{cpu bound}
   \\
   Code optimisé !
  \end{column}
 \end{columns} 

\end{frame}

\begin{frame}[fragile]{Opération matricielle ( suite... )}

\begin{lstlisting}
std::vector<double> A(N*N);
std::vector<double> B(N*N);
std::vector<double> C(N*N);
...
const int szBloc = 127;
// On saucissone les boucles en i et j en plusieurs morceaux 
// de taille szBloc :
for ( int kb = 0; kb < N; kb += szBloc )
  for ( int jb = 0; jb < N; jb += szBloc )
    for ( int ib = 0; ib < N; ib += szBloc )
      for ( int k = kb; k < kb+szBloc; ++k )
        for ( int j = jb; j < jb+szBloc; ++j )
          for ( int i = ib; i < ib+szBloc; ++i )
            C[i+j*N] += A[i+k*N]*B[k+j*N];
\end{lstlisting}

 \begin{columns}[t]
  \begin{column}{0.5\textwidth}
   \textcolor{green!25!black}{Mémoire cache} 
   \\ 
   \textcolor{red}{cpu bound}
   \\
   Code optimisé !
   \end{column}
   \begin{column}{0.5\textwidth}
   \textcolor{orange!25!black}{Mémoire interlacée}
   \\
   \textcolor{blue}{cpu bound}
   \\
   Code un peu moins optimisé !
  \end{column}
 \end{columns}
\end{frame}

\begin{frame}[fragile]{Padding pour la mémoire interlacée}

\begin{lstlisting}
 // Calcul la transposée de la matrice A et la range dans B
 for ( int i = 0; i < N; ++i )
   for ( int j = 0; j < N; ++j ) {
      B[i+j*N] = A[j+i*N];
   }
\end{lstlisting}

\begin{itemize}
\item Accès des données de $A$ optimisé ( lecture linéaire )
\item Pour $B$, plus délicat :
  \begin{itemize}
  \item Si $N$ = multiple du nombre de voies : pire des accès pour $B$ !
  \item Si $N$ = multiple du nombre de voies + 1 : Accès pour $B$ optimal !
  \end{itemize}
\end{itemize}

\end{frame}

\begin{frame}[fragile]{Padding pour la mémoire interlacée}

\begin{lstlisting}
 const int nbMemVoies = ...;
 int padding_N = N+(nbMemVoies + 1 - (N%nbMemVoies))%nbMemVoies;
 std::vector B(N*padding_N);
 ...
 for ( int i = 0; i < N; ++i )
   for ( int j = 0; j < N; ++j ) {
      B[i+j*padding_N] = A[j+i*N];
   }
\end{lstlisting}

\begin{itemize}
 \item Accès optimisé pour $A$ et $B$ !
\end{itemize}

\end{frame}

\section{Utilisation des threads en C++ 2011}

\begin{frame}[fragile]{Exécution et terminaison d'un thread}

Plus simple qu'avec les posix threads :

\begin{lstlisting}
#include <iostream>
#include <thread>

// Cette fonction sera appelée par un thread
void call_from_thread() {
  std::cout << "Hello, World" << std::endl;
}

int main() {
  // Exécution d'un thread
  std::thread t1(call_from_thread);

  //Join the thread with the main thread
  t1.join();

  return 0;
}
\end{lstlisting}

\end{frame}

\begin{frame}[fragile]{Exécution d'un thread avec une lambda fonction}

L'utilisation des fonctions lambdas en C++ 2011 simplifie beaucoup
l'utilisation des threads :

\begin{lstlisting}
#include <iostream>
#include <thread>

int main() {
  // Exécution d'un thread
  std::thread t1([] () { std::cout << "Hello, World" << std::endl; } );
  //Join the thread with the main thread
  t1.join();

  return 0;
}
\end{lstlisting}

\end{frame}

\begin{frame}[fragile]{Exécution d'un groupe de threads}
 
 \begin{lstlisting}
static const int num_threads = 10;

void call_from_thread() {
  std::cout << "Hello, World" << std::endl;
}

int main() {
  std::thread t[num_threads];
  // Exécution d'un groupe de threads
  for (int i = 0; i < num_threads; ++i) {
    t[i] = std::thread(call_from_thread);
  }
  std::cout << "Launched from the main\n";
  //Joint les threads avec le thread principal
  for (int i = 0; i < num_threads; ++i) {
    t[i].join();
  }

  return 0;
}
\end{lstlisting}
\end{frame}

\begin{frame}[fragile]{Lancement d'un thread par CPU}
\begin{lstlisting}
int main(int argc, const char** argv) {
  unsigned num_cpus = std::thread::hardware_concurrency();
  std::cout << "Créer " << num_cpus << " threads\n";
  // mutex permettant un accès ordonnées à std::cout pour les threads
  std::mutex iomutex;
  std::vector<std::thread> threads(num_cpus);
  for (unsigned i = 0; i < num_cpus; ++i) {// Lambda function for thread
    threads[i] = std::thread([&iomutex, i] ()  {
      {
// std::lock_guard permet de bloquer un mutex pendant la durée d'un bloc
// d'instruction. Ici cela permet de bloquer le mutex uniquement durant
// l'affichage à l'aide de std::cout
        std::lock_guard<std::mutex> iolock(iomutex);
        std::cout << "Thread #" << i << " is running\n";
      }
// On simule un travail important en mettant le thread un peu en pause
      std::this_thread::sleep_for(std::chrono::milliseconds(200));
    });
  }
  for (auto& t : threads) { t.join(); }
  return 0;
} 
\end{lstlisting}
\end{frame}

\subsection{Gestion des conflits de données}

\begin{frame}[fragile]{Section protégée par mutex}

{\scriptsize
\begin{itemize}
 \item Pour des sections de code complexes;
 \item Gestion preneuse de ressources CPU;
\end{itemize}
}

\begin{lstlisting}
std::map<std::string, std::string> g_pages;
std::mutex g_pages_mutex;
 
void save_page(const std::string &url) {// simulate a long page fetch
    std::this_thread::sleep_for(std::chrono::seconds(2));
    std::string result = "fake content";
    g_pages_mutex.lock();
    g_pages[url] = result;
    g_pages_mutex.unlock();
}
int main() {
    std::thread t1(save_page, "http://foo");
    std::thread t2(save_page, "http://bar");
    t1.join();  t2.join(); 
    g_pages_mutex.lock();
    for (const auto &pair : g_pages) 
        std::cout << pair.first << " => " << pair.second << '\n';
    g_pages_mutex.unlock();
}
\end{lstlisting}

\end{frame}

\begin{frame}[fragile]{Atomicité}

\begin{itemize}
 \item Pour des opérations simples sur des types de base;
 \item Très léger en ressources CPU.
\end{itemize}

\begin{lstlisting}
# include <atomic>
void compLineMandelbrot( int i, std::vector<unsigned>& img  ) {
// Calcul 1 ligne de l'image de mandelbrot...
}
void compMandelbrot( int W, int H ) {//  mandelbrot multi-threadé :
  volatile std::atomic<int> num_line(-1);
  std::vector<unsigned> img(W*H);
  unsigned num_cpus = std::thread::hardware_concurrency();
  std::vector<std::thread> threads;
  for ( int c = 0; c < num_cpus-1; ++c ) {
    threads.push_back(std::thread([W,H,maxIter,&num_line,&img]()
      {while(num_line<H) {num_line ++;
        compLineMandelbrot(W,H,maxIter,num_line,img); } } ) );
  }
  while ( num_line < H ) {num_line ++;compLineMandelbrot(num_line,img);}
  for ( auto& t : threads ) t.join();
}
\end{lstlisting}

\end{frame}

\begin{frame}[fragile]{Cas d'interblocage avec les mutex}
 
 \begin{lstlisting}
  if ( pid == 0 ) {
    S.lock();
    Q.lock();
    ...
    Q.unlock();
    S.unlock();
  else if ( pid == 1 ) {
    Q.lock();
    S.lock();
    ...
    S.unlock();
    Q.unlock();
  }
 \end{lstlisting}

 Interblocage impossible avec l'atomicité !
 
\end{frame}

\section{OpenMP}

\begin{frame}[fragile]{Vue d'ensemble d'OpenMP}
 
 \begin{block}{Principe}
  \begin{itemize}
   \item Intégré aux compilateurs : option de compilation (\verb@-fopenmp@ ) + directives de compilation
   \lstinline@# pragma omp @;
   \item Définition de régions parallèles avec déclaration variables partagées/privées
   \item Possibilité d'alterner entre régions séquentielles et parallèles.
  \end{itemize}
 \end{block}
 
 \begin{block}{Région parallèle}
  \begin{itemize}
   \item Peut être une boucle parallèle : répartition statique ou dynamique
   \item Plusieurs sections de code en parallèle;
   \item Même section de code exécutée par plusieurs threads.
   \item Contrôle du nombre de thread par directive ou variable d'environnement
   \verb@OMP_NUM_THREADS@
   \item Affinité via variable d'environnement : dépend du compilateur.
  \end{itemize}
 \end{block}
\end{frame}

\subsection{Régions parallèles}

\begin{frame}[fragile]{Déclaration d'une région parallèle}
 \begin{lstlisting}
  # include <iostream>
  # include <omp.h>
  
  int main() {
    float a;
    int p;
    a = 92290; p = 0;
  # pragma omp parallel
    { // Début de la région parallèle
      p = omp_in_parallel();
      std::cout << " a = " << a << " et p = " << p << std::endl;
    }
    return 0;
  }
 \end{lstlisting}
 
 \begin{block}{Sortie du programme}
  \begin{semiverbatim}
   a = 92290 et p = 1
   a = 92290 et p = 1
  \end{semiverbatim}
  Par défaut, une variable est partagée.
 \end{block} 
\end{frame}

\begin{frame}[fragile]{Déclaration d'une région parallèle}
 \begin{lstlisting}
  # include <iostream>
  # include <omp.h>
  
  int main() {
    float a = 92290;
  # pragma omp parallel default(none) private(a) shared(std::cout)
    { // Début de la région parallèle
      a = a + 290;
      std::cout << " a = " << a << std::endl;
    }
    return 0;
  }
 \end{lstlisting}
 
 \begin{block}{Sortie du programme}
  \begin{semiverbatim}
   a = 290
   a = 290
  \end{semiverbatim}
 \end{block} 
\end{frame}

\begin{frame}[fragile]{Déclaration d'une région parallèle}
 \begin{lstlisting}
  # include <iostream>
  # include <omp.h>
  
  int main() {
    float a = 92000;
  # pragma omp parallel default(none) firstprivate(a) shared(std::cout)
    { // Début de la région parallèle
      a = a + 290;
      std::cout << " a = " << a << std::endl;
    }
    return 0;
  }
 \end{lstlisting}
 
 \begin{block}{Sortie du programme}
  \begin{semiverbatim}
   a = 92290
   a = 92290
  \end{semiverbatim}
 \end{block} 
\end{frame}

\begin{frame}[fragile]{\'Etendue d'une région parallèle}
 
\begin{lstlisting}
# include <iostream>
# include <omp.h>

void function() {
    double a = 92290;
    a += omp_get_thread_num();
    std::cout << "a = " << a << std::endl;
}
int main() {
# pragma omp parallel
{
  function();
}
  return 0;
}
\end{lstlisting}

\begin{block}{Sortie du programme}
\begin{semiverbatim}
a = 92290
a = 92291
\end{semiverbatim}
\end{block} 
\end{frame}

\begin{frame}[fragile]{\'Etendue d'une région parallèle}

\begin{lstlisting}
void function(double& a, double& b) {
    b = a + omp_get_thread_num();
    std::cout << "b = " << b << std::endl;
}

int main()
{
  double a = 92290, b;
# pragma omp parallel shared(a) private(b)
  {
      function(a,b);
  }
  return 0;
}
\end{lstlisting}

\begin{block}{Sortie du programme}
\begin{semiverbatim}
b = 92290
b = 92291
\end{semiverbatim}
\end{block}
\end{frame}

\begin{frame}[fragile]{Allocation dynamique et région parallèle}
\begin{lstlisting}
# include <iostream>
# include <omp.h>

int main() {
    int nbTaches, i, deb, fin, rang, n = 1024;
    double* a;
#   pragma omp parallel
    { nbTaches = omp_get_num_threads(); }
    a = new double[n*nbTaches];    
#   pragma omp parallel default(none) private(deb,fin,rang,i) \
                        shared(a,n,std::cout)
    {
        rang = omp_get_thread_num();
        deb  = rang*n; fin = (rang+1)*n-1;
        for ( i = deb; i <= fin; i++ ) a[i] = 92290. + double(i);
        std::cout << "Rang : "<<rang<< " A[" << deb << "] = " << a[deb]
                  << ", A[" << fin << "] = " << a[fin] << std::endl;
    }
    delete [] a;
    return 0;
}
\end{lstlisting}
\end{frame}

\begin{frame}[fragile]{Réduction en OpenMP}

\begin{lstlisting}
 int main() {
    int s = 0;
#   pragma omp parallel default(none) reduction(+:s)
    {
        s = omp_get_thread_num()+1;
    }
    std::cout << "s = " << s << std::endl;
    return EXIT_SUCCESS;
}
\end{lstlisting}

Cette application affichera \verb@s = 3@ sur deux threads.

\end{frame}

\subsection{Division du travail}

\begin{frame}[fragile]{Division des boucles parallèles}

\begin{block}{Parallélisation boucle}
 \begin{itemize}
  \item Se fait à l'aide d'une clause \verb@for@;
  \item Boucles infinies non parallélisables ( \lstinline@do ... while@ );
  \item Trois modes de parallélisation via \verb@schedule@ : Statique, dynamique
  ou guidé ( non abordé ici ).
  \item Par défaut, répartition statique
  \item Synchronisation globale à la fin de la boucle, sauf si clause \verb@nowait@ spécifié.
 \end{itemize}
\end{block}
\end{frame}
\begin{frame}[fragile]{Division des boucles parallèles}

Parallélisation de Mandelbrot avec équilibre des tâches :

\begin{lstlisting}
/* Calcul de l'ensemble de Mandelbrot en OpenMP */
std::vector<int> compMandelbrot( int W, int H, int maxIter ) {
    std::vector<int> img(W*H);
#   pragma omp parallel for schedule(dynamic)
    for ( int i  = 0; i < H; ++i )
        compLineMandelbrot( W,H, maxIter, i, img );
  return img;
}
\end{lstlisting}
\end{frame}

\begin{frame}[fragile]{Parallélisation boucles avec réduction}
\begin{lstlisting}
  const int n = 5;
  int i, s = 0, p = 1, r = 1;
  
# pragma omp parallel for reduction(+:s) reduction(*:p,r)
  for ( i = 1; i <= n; ++i ) {
      s += 1;
      p *= 2;
      r *= 3;
  }
  std::cout << "s = "<<s<< ", p = " <<p<< ", r = " <<r<< std::endl;
\end{lstlisting}

ce qui affichera à l'exécution :

\begin{verbatim}
s = 5, p = 32, r = 243;
\end{verbatim}
\end{frame}

\begin{frame}[fragile]{Sections parallèles}
\begin{lstlisting}
  # pragma omp parallel private(i)
  {
  #  pragma omp sections nowait
    {
  #     pragma omp section 
        for (int i = 0; i < 10; ++i)
          std::cout << "Thread one ! --> " << i << std::endl;
  #     pragma omp section
        for (int i = -10; i < 0; i += 2)
          std::cout << "Thread two ! --> " << i << std::endl;
      }
  }
\end{lstlisting}
\end{frame}

\subsection{Synchronisation}

\begin{frame}[fragile]{clause single}
 
Section uniquement exécutée que par un thread...
 
\begin{lstlisting}
\begin{lstlisting}
#pragma omp parallel default(none) private(a,rang)
{
  a = 92290.;
# pragma omp single
  {
      a = -92290.;
  }
  rang = omp_get_thread_num();
  std::cout << "rang " << rang << " : a = " << a << std::endl;
}
\end{lstlisting}

Sur deux threads :
\begin{verbatim}
rang 0 : a = 92290.
rang 1 : a = -92290.
\end{verbatim}
\end{frame}

\begin{frame}[fragile]{clause master}

Section uniquement exécutée par le thread principal :
\begin{lstlisting}
#pragma omp parallel default(none) private(a,rang)
{
  a = 92290.;
# pragma omp master
  {
      a = -92290.;
  }
  rang = omp_get_thread_num();
  std::cout << "rang " << rang << " : a = " << a << std::endl;
}
\end{lstlisting}

\begin{verbatim}
rang 0 : a = -92290.
rang 1 : a = 92290.
\end{verbatim}
\end{frame}

\begin{frame}[fragile]{Synchronisation}
 
\begin{lstlisting}
  double *a, *b;
  int i, n=5;
# pragma omp parallel
  {
# pragma omp single
    { a = new double[n]; b = new double[n]; }
# pragma omp master
    { for (i = 0; i < n; i++)
        a[i] = (i+1)/2.; }
# pragma omp barrier
# pragma for schedule(static)
    for (i=0; i < n; i++) b[i] = 2.*a[i];
# pragma omp single nowait
    { delete [] a; }
  }
  printf("B equal : ");
  for (i = 0; i < n; i++) printf("%7.5lg\t",b[i]);
  printf("\n");
\end{lstlisting}

\end{frame}

\begin{frame}[fragile]{Atomicité et section critique}

{\scriptsize
Même notions que pour les threads :

Atomicité :
 \begin{lstlisting}
  int counter = 100, rank;
# pragma omp parallel private(rank)
  {
    rank = omp_get_thread_num();
#   pragma omp atomic
    counter += 1;    
    printf("Rank : %d, counter = %d\n",rank,counter);
  }
  printf("Final counter : %d\n",counter);
\end{lstlisting}

Section critique :
\begin{lstlisting}
  int s = 0,p = 1;
# pragma omp parallel
  {
#   pragma omp critical
    {
      s += 1; p *= 2;
    }
  }
  printf ("s = %d, p = %d\n", s, p);
\end{lstlisting}
}

\end{frame}

\section{Autres outils existant}

\subsection{Threading Building BLocks (TBB)}

\begin{frame}[fragile]{Vue d'ensemble}

\begin{itemize}
 \item Bibliothèque template C++ mise en {\oe}uvre par Intel;
 \item License apache;
 \item Boucles parallèles
 \item Algorithmes de réduction;
 \item Gestion graphe de tâches en parallèle;
 \item Gestion concurrence entre threads;
 \item Politique d'allocation pour optimisation cache en multithread;
 \item Conteneurs ( à la STL ) supportant la concurrence.
\end{itemize}

\end{frame}

\begin{frame}[fragile]{Exemple de code TBB}
 
 En reprenant notre éternel Mandelbrot :

\begin{lstlisting}
class CompMandelbrot {
public:
  CompMandelbrot( unsigned W, unsigned H, int maxIter, int* pt_img ) :
    m_W(W), m_H(H), m_maxIter(maxIter), m_pt_img(pt_img)  {}  
  void operator()( const tbb::blocked_range<int>& r ) const {
    double scaleX = 3./(m_W-1);
    double scaleY = 2.25/(m_H-1);
    for ( int i = r.begin(); i != r.end(); ++i ) {
       ...
    } }
private:
  unsigned m_W, m_H, m_maxIter;
  int* m_pt_img;
};
std::vector<int> compMandelbrot( int W, int H, int maxIter ) {
  std::vector<int> img(W*H);
  tbb::parallel_for(tbb::blocked_range<int>(0,H), 
              CompMandelbrot(W,H,maxIter,(int*)img.data()));
  return img;
} 
\end{lstlisting}

\end{frame}

\begin{frame}[fragile]{C++ 2017}
 \begin{block}{Aperçu}
  \begin{itemize}
   \item Parallélisation d'algorithmes de la STL;
   \item Dans la norme, pas besoin d'outils externes !
  \end{itemize}

 \end{block}

\begin{lstlisting}
std::vector<int> v = ...
std::sort(vec.begin(), vec.end());// standard sequential sort
using namespace std::experimental::parallel;
sort(seq, v.begin(), v.end());// explicitly sequential sort
sort(par, v.begin(), v.end());// permitting parallel execution
sort(vec, v.begin(), v.end());// permitting vectorization as well
// sort with dynamically-selected execution
size_t threshold = ...
execution_policy exec = seq;
if(v.size() > threshold)
{
exec = par;
}
sort(exec, v.begin(), v.end());
\end{lstlisting}
\end{frame}

\end{document}
